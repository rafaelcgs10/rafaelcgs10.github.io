% Created 2021-07-09 sex 15:36
% Intended LaTeX compiler: pdflatex
\documentclass[presentation]{beamer}
\usepackage[utf8]{inputenc}
\usepackage[T1]{fontenc}
\usepackage{graphicx}
\usepackage{grffile}
\usepackage{longtable}
\usepackage{wrapfig}
\usepackage{rotating}
\usepackage[normalem]{ulem}
\usepackage{amsmath}
\usepackage{textcomp}
\usepackage{amssymb}
\usepackage{capt-of}
\usepackage{hyperref}
\usepackage{graphicx, hyperref, url}
\mode<beamer>{\usetheme{Madrid}}
\usetheme{default}
\author{Rafael Castro - rafaelcgs10.github.io}
\date{14/07/2021}
\title{BBL sobre Nix}
\hypersetup{
 pdfauthor={Rafael Castro - rafaelcgs10.github.io},
 pdftitle={BBL sobre Nix},
 pdfkeywords={},
 pdfsubject={},
 pdfcreator={Emacs 27.2 (Org mode 9.4.6)}, 
 pdflang={Portuguese}}
\begin{document}

\maketitle

\section{Introdução}
\label{sec:org1a7a150}

\begin{frame}[label={sec:org8cac701}]{O que é cada coisa?}
\begin{itemize}
\item Nix: Um gerenciador de pacotes \alert{puramente funcional}
\item Nix Expression Language: Uma linguagem pura, funcional e lazy
\item NixOS: Uma distro Linux feita envolta do Nix
\item Home Manager: Um Manager (gerenciador) para espaço de usuário envolta do Nix
\end{itemize}
\end{frame}

\section{Show of:}
\label{sec:org56f994e}

\begin{frame}[label={sec:org4da9829}]{Instalando o meu ambiente numa VM de forma semi-automática}
\end{frame}

\section{Como usar Nix?}
\label{sec:org9e199a3}

\begin{frame}[label={sec:orgd468595}]{Basta instalar e usar!}
\end{frame}

\begin{frame}[label={sec:org8f6df4d}]{Também podemos ter arquivos .nix para definirmos nossos shells.}
\end{frame}

\section{Como usar Home Manager?}
\label{sec:org7b50b4d}

\begin{frame}[label={sec:org79836db}]{Basta editar os arquivos .nix}
\end{frame}

\section{Como usar NixOS?}
\label{sec:orgfcae63d}

\begin{frame}[label={sec:org55df5e7}]{Basta editar os arquivos .nix em /etc/nix!}
\end{frame}

\section{Nix vs Docker}
\label{sec:org88c08b1}

\begin{frame}[label={sec:org0d8f2d7}]{Nix não é um substituto para o Docker, ele faz coisas a mais e coisas a menos.}
\end{frame}
\end{document}
