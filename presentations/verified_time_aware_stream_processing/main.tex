\documentclass[aspectratio=169,10pt]{beamer}
\usetheme{Madrid}
\usepackage[T1]{fontenc}

\usepackage{fancybox,graphicx,hyperref,url}
\usepackage{tikz}
\usetikzlibrary{shapes,arrows}
\usetikzlibrary{positioning}
\usepackage{booktabs}
\usepackage{enumitem}

\usepackage{listings}
\usepackage{pdfpages}
\usepackage{lstautogobble}
\input{lstisabelle}
\usepackage[listings,skins,breakable,xparse]{tcolorbox}
\tcbuselibrary{theorems}
\tcbset{highlight math/.append style={boxrule=0pt,
                                      frame hidden,
                                      colback=yellow!40!white,
                                      sharp corners}}

\usepackage{xpatch}
\usepackage{xcolor}
\usepackage{realboxes}
\usetikzlibrary{fit}
\usetikzlibrary{shadings}
\usetikzlibrary{shapes.arrows,shadows.blur}

\pgfdeclarefunctionalshading{Hermite-Gaussian modes}{\pgfpoint{-25bp}{-25bp}}{\pgfpoint{25bp}{25bp}}{}{
    10 atan sin 1000 mul cos 1 add
    exch
    10 atan sin 1000 mul cos 1 add
    mul 4 div
    dup dup
}

\makeatletter
\xpretocmd\lstinline{\Colorbox{yellow!10!white}\bgroup\appto\lst@DeInit{\egroup}}{}{}
\makeatother

\definecolor{my_red}{RGB}{128, 0, 0}

\lstset{captionpos=b}
\lstset{numberbychapter=false}
\lstset{autogobble}
% \lstset{breaklines=true}

\usepackage{tikz}
\usepackage{subcaption}
\usetikzlibrary{calc, chains, decorations.pathmorphing}
\usetikzlibrary{shapes,arrows,backgrounds}
\usetikzlibrary{positioning,fit,shapes.geometric,shapes}

\setbeamercovered{transparent}

\setlistdepth{9}
\setlist[itemize,1]{label=$\bullet$}
\setlist[itemize,2]{label=$\bullet$}
\setlist[itemize,3]{label=$\bullet$}
\setlist[itemize,4]{label=$\bullet$}
\setlist[itemize,5]{label=$\bullet$}
\setlist[itemize,6]{label=$\bullet$}
\setlist[itemize,7]{label=$\bullet$}
\setlist[itemize,8]{label=$\bullet$}
\setlist[itemize,9]{label=$\bullet$}
\renewlist{itemize}{itemize}{9}

\setlist[enumerate,1]{label=$\arabic*.$}
\setlist[enumerate,2]{label=$\alph*.$}
\setlist[enumerate,3]{label=$\roman*.$}
\setlist[enumerate,4]{label=$\arabic*.$}
\setlist[enumerate,5]{label=$\alpha*$}
\setlist[enumerate,6]{label=$\roman*.$}
\setlist[enumerate,7]{label=$\arabic*.$}
\setlist[enumerate,8]{label=$\alph*.$}
\setlist[enumerate,9]{label=$\roman*.$}
\renewlist{enumerate}{enumerate}{9}

\AtBeginSection[]{
  \begin{frame}[noframenumbering]
    \vfill
    \centering
    \begin{beamercolorbox}[sep=8pt,center,shadow=true,rounded=true]{title}
      \usebeamerfont{title}\insertsectionhead\par%
    \end{beamercolorbox}
    \vfill
  \end{frame}
}

\title[Time-Aware Stream Processing in Isabelle/HOL]{Time-Aware Stream Processing in Isabelle/HOL}

\author[Rafael Castro and Dmitriy Traytel]{
  Rafael Castro G. Silva and  Dmitriy Traytel\\\medskip
  {\small \url{rasi@di.ku.dk}}}

\date{14/09/2024}

% change title (done)
% (co)datatypes -> codatatypes (done)
% while_option from HOL library (done)
% coindtive principle -> coinduction principle (done)
% remove next step slide, andd as a bullet (done)
% explain slice in more detail (done)
% mention selectors in llist slide, drop lmap, drop lset (done)
% move count operator example to slide that introduce op type (done)
% modeling lazy list processors which do not care about time (done)
% reduce Corecursion is like recursion line (done)
% why partial order? mention timely dataflow (done)
% explain what incremental, and set expectation for example (done)
% move arrows (done)
% Change push to out, done to out (done)
% put star footnote on op type saying it is simplified compare to the paper, also in the composition lemma (done)
% write the lapp lemma about lfinite (done)
% show coinduction principle for llist (done)
% add link llist afp (done)
% on produce slide, metion lshift as friend (point to freinds with benefits paper) (done)
% justify the compositional reasoning with lines of code for each building block (done)
% slide about tecnicques used, coinduction principles, friends, reasoning up to friends, generalization using skip_op, combining coinduction and induction, coinduction up to congruence for coinductive predicates, say our paper is more detailed in these topics
%
% add definition of batch_op, and all 4 properties

\institute[UCPH]{
  Department of Computer Science \\
  University of Copenhagen}

\begin{document}
\setbeamercovered{invisible}
% \setbeamercovered{dynamic}

\begin{frame}
  \titlepage

\end{frame}
\section{Introduction}

\begin{frame}[fragile]
  \frametitle{Time-Aware Stream Processing}
  \begin{itemize}
    \item What is Time-Aware Stream Processing?
          \begin{itemize}
                  \pause
            \item Stream Processing: programs that compute (possibly) unbounded sequences of data (streams)
                  \pause
            \item Time-Aware: Data has timestamp metadata; timestamps are bounded by watermarks
                  \begin{itemize}
                    \item Timestamp: A partially-ordered value associated with the data (e.g., unix-time, int, pairs of ints, etc...)
                    \item Watermark: A value of the same type of the timestamp. Represents data-completeness.
                          \begin{figure}
                            \centering
                            \includegraphics[width=.75\textwidth]{stream_ex1.pdf}
                          \end{figure}
                  \end{itemize}
          \end{itemize}
          \pause
    \item Asynchronous Dataflow Programming: Directed graph of interconnected operators that perform event-wise transformations
    \item E.g.: Apache Flink, Apache Samza, Apache Spark, Google Cloud Dataflow, and Timely Dataflow
          \vspace*{-1ex}
          \begin{overlayarea}{\textwidth}{.1\textheight}
            \centering
            \begin{figure}
              \centering
              \only<2->{\includegraphics[scale=0.1]{all.png}}
            \end{figure}
          \end{overlayarea}
          \vspace*{-1ex}
          \pause
    \item Why?
          \begin{itemize}
            \item Highly Parallel
            \item Low latency (output as soon as possible)
            \item Incremental computing (re-uses previous computations)
          \end{itemize}
  \end{itemize}
\end{frame}

\begin{frame}[fragile]
  \frametitle{Time-Aware Stream Processing Example}
  Example:\\
  \textbf{Incrementally} count the occurrences of the words ``dog'' and ``cat''
  \begin{figure}
    \centering
    \includegraphics[width=.75\textwidth]{stream_ex1.pdf}
  \end{figure}
\end{frame}

\begin{frame}[fragile]
  \frametitle{Time-Aware Stream Processing Example}
    \includegraphics[page=1,width=.95\textwidth]{dataflow_ex1.pdf}
\end{frame}

\begin{frame}[fragile,noframenumbering]
  \frametitle{Time-Aware Stream Processing Example}
    \includegraphics[page=2,width=.95\textwidth]{dataflow_ex1.pdf}
\end{frame}

\begin{frame}[fragile,noframenumbering]
  \frametitle{Time-Aware Stream Processing Example}
    \includegraphics[page=3,width=.95\textwidth]{dataflow_ex1.pdf}
\end{frame}

\begin{frame}[fragile,noframenumbering]
  \frametitle{Time-Aware Stream Processing Example}
    \includegraphics[page=4,width=.95\textwidth]{dataflow_ex1.pdf}
\end{frame}

\begin{frame}[fragile,noframenumbering]
  \frametitle{Time-Aware Stream Processing Example}
    \includegraphics[page=5,width=.95\textwidth]{dataflow_ex1.pdf}
\end{frame}

\begin{frame}[fragile,noframenumbering]
  \frametitle{Time-Aware Stream Processing Example}
    \includegraphics[page=6,width=.95\textwidth]{dataflow_ex1.pdf}
\end{frame}

\begin{frame}[fragile,noframenumbering]
  \frametitle{Time-Aware Stream Processing Example}
    \includegraphics[page=7,width=.95\textwidth]{dataflow_ex1.pdf}
\end{frame}

\begin{frame}[fragile,noframenumbering]
  \frametitle{Time-Aware Stream Processing Example}
    \includegraphics[page=8,width=.95\textwidth]{dataflow_ex1.pdf}
\end{frame}

\begin{frame}[fragile,noframenumbering]
  \frametitle{Time-Aware Stream Processing Example}
    \includegraphics[page=9,width=.95\textwidth]{dataflow_ex1.pdf}
\end{frame}

\begin{frame}[fragile,noframenumbering]
  \frametitle{Time-Aware Stream Processing Example}
    \includegraphics[page=10,width=.95\textwidth]{dataflow_ex1.pdf}
\end{frame}

\begin{frame}[fragile,noframenumbering]
  \frametitle{Time-Aware Stream Processing Example}
    \includegraphics[page=11,width=.95\textwidth]{dataflow_ex1.pdf}
\end{frame}

\begin{frame}[fragile,noframenumbering]
  \frametitle{Time-Aware Stream Processing Example}
    \includegraphics[page=12,width=.95\textwidth]{dataflow_ex1.pdf}
\end{frame}

\begin{frame}[fragile,noframenumbering]
  \frametitle{Time-Aware Stream Processing Example}
    \includegraphics[page=13,width=.95\textwidth]{dataflow_ex1.pdf}
\end{frame}

\begin{frame}[fragile,noframenumbering]
  \frametitle{Time-Aware Stream Processing Example}
    \includegraphics[page=14,width=.95\textwidth]{dataflow_ex1.pdf}
\end{frame}

\begin{frame}[fragile,noframenumbering]
  \frametitle{Time-Aware Stream Processing Example}
    \includegraphics[page=15,width=.95\textwidth]{dataflow_ex1.pdf}
\end{frame}

\begin{frame}[fragile,noframenumbering]
  \frametitle{Time-Aware Stream Processing Example}
    \includegraphics[page=16,width=.95\textwidth]{dataflow_ex1.pdf}
\end{frame}

\begin{frame}[fragile,noframenumbering]
  \frametitle{Time-Aware Stream Processing Example}
    \includegraphics[page=17,width=.95\textwidth]{dataflow_ex1.pdf}
\end{frame}

\begin{frame}[fragile,noframenumbering]
  \frametitle{Time-Aware Stream Processing Example}
    \includegraphics[page=18,width=.95\textwidth]{dataflow_ex1.pdf}
\end{frame}

\begin{frame}[fragile,noframenumbering]
  \frametitle{Time-Aware Stream Processing Example}
    \includegraphics[page=19,width=.95\textwidth]{dataflow_ex1.pdf}
\end{frame}

\begin{frame}[fragile,noframenumbering]
  \frametitle{Time-Aware Stream Processing Example}
    \includegraphics[page=20,width=.95\textwidth]{dataflow_ex1.pdf}
\end{frame}

\begin{frame}[fragile,noframenumbering]
  \frametitle{Time-Aware Stream Processing Example}
    \includegraphics[page=21,width=.95\textwidth]{dataflow_ex1.pdf}
\end{frame}

\begin{frame}[fragile,noframenumbering]
  \frametitle{Properties}
  \begin{itemize}
    \item How do we know if our Dataflow program is what we want?
          \pause
    \item We need a correctness specification
          \pause
    \item \textbf{Intuition of the specification}:
          \begin{itemize}
            \item Soundness: for every output \textit{DT t H}, the ``dog'' count in \textit{H} is the count of events with timestamp $(\le) t$ which contains the string ``dog''; similarly for ``cat''. The count for any other word is always 0.
            \item Completeness: The other way around.
          \end{itemize}
    % \item What is correctness for a program that does not always terminate?
  \end{itemize}
  \begin{figure}
    \centering
    \includegraphics[width=.75\textwidth]{correctness.pdf}
  \end{figure}
\end{frame}

\begin{frame}[fragile,noframenumbering]
  \frametitle{How to prove it}
  \begin{itemize}
    \item \textbf{lets break down the problem!}:
          \begin{itemize}
            \item The correctness of the entire Dataflow emerges from the correctness of each part (operator)
                  \begin{itemize}
                    \item Operator 1: Slicer
                    \item Operator 2: Filter
                    \item Operator 3: Incremental histogram
                          \begin{itemize}
                            \item Assumptions about the incoming stream:
                                  \begin{enumerate}
                                    \item Monotone: after \text{WM wm} no \textit{DT t d} such that $t \le wm$.
                                    \item Productive: after \textit{DT t d} eventually \text{WM wm} such that $t \le wm$
                                  \end{enumerate}
                          \end{itemize}
                  \end{itemize}
                  \includegraphics[width=.6\textwidth]{dataflow_ex1_b.pdf}
          \end{itemize}
    \item The original incoming stream must respect monotonicity and productivity
          \begin{figure}
            \centering
            \includegraphics[width=.70\textwidth]{stream_ex1.pdf}
          \end{figure}
    \item Each operator must preserve monotonicity and productivity!
  \end{itemize}
\end{frame}

\section{Formalization in Isabelle/HOL}

% \begin{frame}[fragile]
%   \frametitle{Isabelle/HOL}
%   \begin{itemize}
%     \item Classical higher-order logic (HOL): Simple Typed Lambda Calculus + (Hilbert) axiom of choice + axiom of infinity + rank-1 polymorphism
%           \pause
%     \item Isabelle: A generic proof assistant
%           \begin{overlayarea}{\textwidth}{.45\textheight}
%             \centering
%             \begin{figure}
%               \centering
%               \only<2>{\includegraphics[scale=0.15]{isabelle}}
%             \end{figure}
%           \end{overlayarea}
%     \item Isabelle/HOL: Isabelle's flavor of HOL
%   \end{itemize}
% \end{frame}

\begin{frame}[fragile]
  \frametitle{Isabelle/HOL: Codatatypes}
  \begin{itemize}
    \item Codatatypes
          \vspace*{-1ex}
          \begin{tcblisting}{hbox,listing only,listing options={language=isabelle,aboveskip=0pt,belowskip=0pt},size=fbox,boxrule=0pt,frame hidden,arc=0pt,colback=yellow!10!white}
codatatype llist = (lnull: LNil) | LCons (lhd: 'a) (ltl: "'a llist")
          \end{tcblisting}
          \vspace*{-1ex}
    \item Selectors: \textit{lhd}, \textit{ltl}, Discriminator: \textit{lnull}
    \item Examples:
          \begin{itemize}
            \item \is{LNil}
            \item \is{LCons 1 (LCons 2 (LCons 3 LNil))}
            \item \is{LCons 0 (LCons 0 (LCons 0 (...)))}
          \end{itemize}
          \vspace*{-1ex}
          \pause
    \item Coinduction principle for lazy list equality
          \begin{tcblisting}{hbox,listing only,listing options={language=isabelle,aboveskip=0pt,belowskip=0pt},size=fbox,boxrule=0pt,frame hidden,arc=0pt,colback=yellow!10!white}
R lxs lys ==>
(!! lxs lys. R lxs lys ==>
  lnull lxs = lnull lys \and
  (\<not> lnull lxs --> \<not> lnull lys --> lhd lxs = lhd lys \and R (ctl lxs) (ctl lys))) ==>
lxs = lys
\end{tcblisting}
    \item More about \textit{llist} at the \textbf{Coinductive} AFP entry!
  \end{itemize}
  % \vspace*{-1ex}
  % \begin{overlayarea}{\textwidth}{.45\textheight}
  %   \begin{figure}
  %     \centering
  %     \only<3>{\includegraphics[scale=0.3]{equality_1.png}}
  %   \end{figure}
  % \end{overlayarea}
\end{frame}


% \begin{frame}[fragile,noframenumbering]
%   \frametitle{Isabelle/HOL: (Co)datatypes}
%   \begin{itemize}
%     \item Datatypes and Codatatypes
%           \vspace*{-1ex}
%           \begin{tcblisting}{hbox,listing only,listing options={language=isabelle,aboveskip=0pt,belowskip=0pt},size=fbox,boxrule=0pt,frame hidden,arc=0pt,colback=yellow!10!white}
%             codatatype (lset: 'a) llist = lnull: LNil | LCons (lhd: 'a) (ltl: "'a llist")
%             for map: lmap where "ltl LNil = LNil"
%           \end{tcblisting}
%           \vspace*{-1ex}
%     \item Examples:
%           \begin{itemize}
%             \item \is{LNil}
%             \item \is{LCons 1 (LCons 2 (LCons 3 LNil))}
%             \item \is{LCons 0 (LCons 0 (LCons 0 (...)))}
%           \end{itemize}
%           \vspace*{-1ex}
%     \item Coinductive principle for lazy list equality:
%           \begin{itemize}
%             \item Show that there is a ``pair of goggles '' (relation) that makes them to look the same:
%                   \begin{itemize}
%                     \item The first lazy list is empty iff second is
%                     \item They have the same head
%                     \item Their tail looks the same
%                   \end{itemize}
%           \end{itemize}
%   \end{itemize}
%   \vspace*{-1ex}
%   \begin{overlayarea}{\textwidth}{.45\textheight}
%     \begin{figure}
%       \centering
%       \includegraphics[scale=0.3]{equality_2.png}
%     \end{figure}
%   \end{overlayarea}
% \end{frame}

% % TODO add notes for explaining codatatype command
% \begin{frame}[fragile,noframenumbering]
%   \frametitle{Isabelle/HOL: (Co)datatypes}
%   \begin{itemize}
%     \item Datatypes and Codatatypes
%           \vspace*{-1ex}
%           \begin{tcblisting}{hbox,listing only,listing options={language=isabelle,aboveskip=0pt,belowskip=0pt},size=fbox,boxrule=0pt,frame hidden,arc=0pt,colback=yellow!10!white}
%             codatatype (lset: 'a) llist = lnull: LNil | LCons (lhd: 'a) (ltl: "'a llist")
%             for map: lmap where "ltl LNil = LNil"
%           \end{tcblisting}
%           \vspace*{-1ex}
%     \item Examples:
%           \begin{itemize}
%             \item \is{LNil}
%             \item \is{LCons 1 (LCons 2 (LCons 3 LNil))}
%             \item \is{LCons 0 (LCons 0 (LCons 0 (...)))}
%           \end{itemize}
%           \vspace*{-1ex}
%     \item Coinductive principle for lazy list equality:
%           \begin{itemize}
%             \item Show that there is a ``pair of goggles '' (relation) that makes them to look the same:
%                   \begin{itemize}
%                     \item The first lazy list is empty iff second is
%                     \item They have the same head
%                     \item Their tail looks the same
%                   \end{itemize}
%           \end{itemize}
%   \end{itemize}
%   \vspace*{-1ex}
%   \begin{overlayarea}{\textwidth}{.45\textheight}
%     \begin{figure}
%       \centering
%       \includegraphics[scale=0.3]{equality.png}
%     \end{figure}
%   \end{overlayarea}
% \end{frame}

\begin{frame}[fragile]
  \frametitle{Isabelle/HOL: Recursion and While Combinator}
  \begin{itemize}
    \item Recursion
          \begin{tcblisting}{hbox,listing only,listing options={language=isabelle,aboveskip=0pt,belowskip=0pt},size=fbox,boxrule=0pt,frame hidden,arc=0pt,colback=yellow!10!white}
fun lshift :: "'a list => 'a llist => 'a llist" (infixr @@ 65) where
  "lshift [] lxs = lxs"
| "lshift (x # xs) lxs = LCons x (lshift xs lxs)"
          \end{tcblisting}
    \item While Combinator
          \begin{tcblisting}{hbox,listing only,listing options={language=isabelle,aboveskip=0pt,belowskip=0pt},size=fbox,boxrule=0pt,frame hidden,arc=0pt,colback=yellow!10!white}
definition while_option :: "('a => bool) => ('a => 'a) => 'a => 'a option" where
"while_option b c s = $\ldots$"
          \end{tcblisting}
          \begin{itemize}
            \item From HOL-Library
            \item \textit{None} (Diverged), \textit{Some x} (Finished with final state \textit{x})
    \item While rule for invariant reasoning (Hoare-style):
          \begin{itemize}
            \item There is something that holds before a step; that thing still holds after the step
          \end{itemize}
          \end{itemize}
  \end{itemize}
\end{frame}

\begin{frame}[fragile]
  \frametitle{Isabelle/HOL: Corecursion}
  \begin{itemize}
    \item Recursion: always eventually reduces an argument
    \item Corecursion: always eventually produces something
          \pause
    \item Corec:
          \vspace*{-1ex}
          \begin{tcblisting}{hbox,listing only,listing options={language=isabelle,aboveskip=0pt,belowskip=0pt},size=fbox,boxrule=0pt,frame hidden,arc=0pt,colback=yellow!10!white}
corec lapp :: "'a llist => 'a llist => 'a llist" where
"lapp lxs lys = case lxs of LNil => lys | LCons x lxs' => LCons x (lapp lxs' lys)"

\<not> lfinite lxs ==> lapp lxs lys = lxs
          \end{tcblisting}
          \vspace*{-1ex}
          % \pause
    % \item Friendly function
    %       \begin{itemize}
    %         \item Preserves productivity: it may consume at most one constructor to produce one constructor.
    %         \item \is{lshift} (\is{@@}) is proved to a be friend
    %       \end{itemize}
    %       \pause
    % \item Coinduction up to congruence: Coinduction for Lazy list equality can be extended to compare an entire finite prefix through a congruence relation
  \end{itemize}
\end{frame}

\begin{frame}[fragile]
  \frametitle{Isabelle/HOL: (Co)inductive Predicates}
  \begin{itemize}
    \item Inductive predicate
          \begin{itemize}
            \item Finite number of introduction rule applications
          \end{itemize}
          \vspace*{-1ex}
          \begin{tcblisting}{hbox,listing only,listing options={language=isabelle,aboveskip=0pt,belowskip=0pt},size=fbox,boxrule=0pt,frame hidden,arc=0pt,colback=yellow!10!white}
inductive in_llist :: "'a => 'a llist => bool" where
  In_llist: "in_llist x (LCons x lxs)"
| Next_llist: "in_llist x lxs => in_llist x (LCons y lxs)"

in_llist 2 (LCons 1 (LCons (2 (...))))
          \end{tcblisting}
          \vspace*{-1ex}
          \pause
    \item Coinductive predicate
          \begin{itemize}
            \item Infinite number of introduction rule applications
          \end{itemize}
          \vspace*{-1ex}
          \begin{tcblisting}{hbox,listing only,listing options={language=isabelle,aboveskip=0pt,belowskip=0pt},size=fbox,boxrule=0pt,frame hidden,arc=0pt,colback=yellow!10!white}
coinductive lprefix :: "'a llist => 'a llist => bool" where
  LNil_lprefix: "lprefix LNil lys"
| LCons_lprefix: "lprefix lxs lys => lprefix (LCons x lxs) (LCons x lys)"

lprefix (LCons 1 (LCons (2 (...)))) (LCons 1 (LCons (2 (...))))
          \end{tcblisting}
          \vspace*{-1ex}
          \pause
    \item Coinduction principle
    % \item But not coinduction up to congruence for free
  \end{itemize}
\end{frame}

\section{Lazy Lists Processors (a.k.a. Non-Time-Aware Stream Processing)}

\begin{frame}[fragile]
  \frametitle{Operator formalization}
  \begin{itemize}
    \item Operator as a codatatype\footnote{This is a simplification of the codatatype in the paper!}

          \begin{itemize}
            \item Taking \is{'i} as the input type, and \is{'o} as the output type:
                  \vspace*{-1.5ex}
                  \begin{tcblisting}{hbox,listing only,listing options={language=isabelle,aboveskip=0pt,belowskip=0pt},size=fbox,boxrule=0pt,frame hidden,arc=0pt,colback=yellow!10!white}
codatatype ('o, 'i) op = Logic ("apply": "('i => ('o, 'i) op $\times$ 'o list)")
                  \end{tcblisting}

                  \vspace*{-1.5ex}
                  \pause
            \item Infinite trees: applying the selector \is{apply} ``walks'' a branch of the tree
          \end{itemize}
\begin{tcblisting}{hbox,listing only,listing options={language=isabelle,aboveskip=0pt,belowskip=0pt},size=fbox,boxrule=0pt,frame hidden,arc=0pt,colback=yellow!10!white}
corec count_op where "count_op P n =
  Logic ($\lambda$e. if P e then (count_op P (n + 1), [n+1]) else (count_op P n, []))"
\end{tcblisting}

\begin{tcblisting}{hbox,listing only,listing options={language=isabelle,aboveskip=0pt,belowskip=0pt},size=fbox,boxrule=0pt,frame hidden,arc=0pt,colback=yellow!10!white}
apply (count_op is_even 0) 0 = (count_op is_even 1, [1])
\end{tcblisting}
  \end{itemize}
\end{frame}

% TODO add some figure showing the tree
\begin{frame}[fragile]
  \frametitle{Execution formalization}
  \begin{itemize}
    \item Produce function: applies the logic (co)recursively throughout a lazy list
          \vspace*{-1.5ex}
          \begin{tcblisting}{hbox,listing only,listing options={language=isabelle,aboveskip=0pt,belowskip=0pt},size=fbox,boxrule=0pt,frame hidden,arc=0pt,colback=yellow!10!white}
definition "produce_1 op lxs = while_option $\ldots$

corec produce where
"produce op lxs = (case produce_1 op lxs of
  None => LNil
| Some (op', x, xs, lxs') => LCons x (xs @@ produce op' lxs'))"
          \end{tcblisting}
          \vspace*{-1.5ex}
    \item \text{lshift (@@)} is a friend of corec function!\footnote{More about it at the ``Friends with benefits'' paper by Jasmin Christian Blanchette, Aymeric Bouzy, Andreas Lochbihler, Andrei Popescu, and
Dmitriy Traytel}
          \pause
    \item \is{produce_1} has an induction principle based on the while invariant rule
  \end{itemize}
\end{frame}

\begin{frame}[fragile]
  \frametitle{Operators: Count}
  \begin{itemize}
    \item Example:
\begin{tcblisting}{hbox,listing only,listing options={language=isabelle,aboveskip=0pt,belowskip=0pt},size=fbox,boxrule=0pt,frame hidden,arc=0pt,colback=yellow!10!white}
corec count_op where "count_op P n =
  Logic ($\lambda$e. if P e then (count_op P (n + 1), [n+1]) else (count_op P n, []))"
\end{tcblisting}
  \end{itemize}
  \begin{figure}[!t]
    \centering
    \begin{tikzpicture}[scale=0.9, every node/.style={scale=0.9}]
      \begin{scope}[local bounding box=scope1]
        \node[] (0,0) (s1) {\text{\is{stream_1}}};
        \node[right = 0.0cm of s1] (eq) {$=$};
      \end{scope}
      \begin{scope}[shift={($(scope1.east)+(0.8cm,0)$)}]
        \tikzset{tape/.style={minimum size=.6cm, draw}}
        \begin{scope}[start chain=0 going right, node distance=0mm]
          \foreach \x [count=\i] in {\is{0},\is{3},\is{3},\is{6}, \is{24}} {
            \node [on chain=0, tape] (a\i) {\x};
          }
        \end{scope}
      \end{scope}

      \begin{scope}[shift={($(scope1.center)+(-2.2cm,-1cm)$)},local bounding box=scope3]
        \node[] (0,0) (prod) {\text{\is{produce (count_op is_even 0)}}};
        \node[right = 0.0cm of prod] (s1) {\text{\is{stream_1}}};
        \node[right = 0.0cm of s1] (eq) {$=$};
      \end{scope}
      \begin{scope}[shift={($(scope3.east)+(0.9cm,0)$)}]
        \tikzset{tape/.style={minimum size=.6cm, draw}}
        \begin{scope}[start chain=0 going right, node distance=0mm]
          \foreach \x [count=\i] in {\is{1},\is{2}, \is{3}} {
            \node [on chain=0, tape] (b\i) {\x};
          }
        \end{scope}
      \end{scope}
      \draw[->, thick]  (a1.south) -- (b1.north);
      \draw[->, thick]  (a4.south) -- (b2.north);
      \draw[->, thick]  (a5.south) -- (b3.north);
    \end{tikzpicture}
  \end{figure}
\end{frame}


\begin{frame}[fragile]
  \frametitle{Sequential Composition operator}
  \begin{itemize}
    \item Sequential composition: take the output of the first operator and give it as input to the second operator.
    \item Correctness\footnote{This is a simplification of the lemma in the paper!}:
          \vspace*{-1ex}
          \begin{tcblisting}{hbox,listing only,listing options={language=isabelle,aboveskip=0pt,belowskip=0pt},size=fbox,boxrule=0pt,frame hidden,arc=0pt,colback=yellow!10!white}
"produce (comp_op op_1 op_2) lxs = produce op_2 (produce op_1 lxs)"
          \end{tcblisting}
          \vspace*{-1ex}
    \item Proof: coinduction principle for lazy list equality and \is{produce_1} induction principle
  \end{itemize}
\end{frame}

\section{Time-Aware Operators}

\begin{frame}[fragile]
  \frametitle{Time-Aware Streams}
  \begin{itemize}
    \item Time-Aware lazy lists
          \vspace*{-1ex}
          \begin{tcblisting}{hbox,listing only,listing options={language=isabelle,aboveskip=0pt,belowskip=0pt},size=fbox,boxrule=0pt,frame hidden,arc=0pt,colback=yellow!10!white}
datatype ('t::order, 'd) event = DT (tmp: 't) (data: 'd) | WM (wmk: 't)
          \end{tcblisting}
          \vspace*{-1ex}
          \pause
    \item Generalization to partial orders
          \begin{itemize}
            \item Cycles
                  \begin{itemize}
                    \item Timely Dataflow: uses pairs to count data cycles (e.g. \textit{(data round n, at cycle i)})
                  \end{itemize}
            \item Operators with multiple inputs
                  \begin{itemize}
                    \item sum type: represents the multiple input/output ports
                  \end{itemize}
          \end{itemize}
    \item Productive and monotone streams: Coinductive predicates over lazy lists of events.
  \end{itemize}
\end{frame}

\begin{frame}[fragile]
  \frametitle{Proving histogram correct: building Blocks}
  \begin{itemize}
    \item Histogram operator: batching and incremental computing
    \item Building Blocks: reusable operators
          \begin{itemize}
            \item Batching: \text{batch\_op}
            \item Incremental computing: \text{incr\_op}
            \item Soundness, completeness, preservation of monotonicity and productivity
          \end{itemize}
          \item Define histogram using the building blocks
          \item Compositional Reasoning: correctness follows from the correctness of the building blocks
          \begin{itemize}
            \item Building blocks: more than 9000 loc
            \item Incremental Histogram: 300 loc
          \end{itemize}
  \end{itemize}
\end{frame}

% \begin{frame}[fragile]
%   \frametitle{Compositional Reasoning}
%   \begin{itemize}
%     \item \is{batch_op} and \is{incr_op} can be composed
%           \vspace*{-1ex}
%           \begin{tcblisting}{hbox,listing only,listing options={language=isabelle,aboveskip=0pt,belowskip=0pt},size=fbox,boxrule=0pt,frame hidden,arc=0pt,colback=yellow!10!white}
% definition "incr_batch_op buf1 buf2 = comp_op (batch_op buf1) (incr_op buf2)"
%           \end{tcblisting}
%           \vspace*{-1ex}
%     \item Soundness, completeness, and monotone and productive preservation
%   \end{itemize}
% \end{frame}

\begin{frame}[fragile]
  \frametitle{Batching Operator}
  The \textit{batch\_op} operator\footnote{Another simplification from the paper}:
  \vspace*{-1ex}
  \begin{tcblisting}{hbox,listing only,listing options={language=isabelle,aboveskip=0pt,belowskip=0pt},size=fbox,boxrule=0pt,frame hidden,arc=0pt,colback=yellow!10!white}
fun batches where "batches [] tds = ([], tds)"
|  "batches (wm # wms) tds = let (bs, tds') = batches wms [(t, d) $\leftarrow$ tds. $\neg$ t <= wm]
                                    in (DT wm [(t, d) $\leftarrow$ tds. t <= wm] # bs, tds')"

corec batch_op where
  "batch_op tds = Logic ($\lambda$ev. case ev of DT t d => (batch_op (tds @ [(t, d)]), [])
  | WM wm => let (out, tds') = batches [wm] tds
                   in (batch_op tds', [x $\leftarrow$ out. data x $\ne$ []] @ [WM wm]))"
\end{tcblisting}

\begin{tcblisting}{hbox,listing only,listing options={language=isabelle,aboveskip=0pt,belowskip=0pt},size=fbox,boxrule=0pt,frame hidden,arc=0pt,colback=yellow!10!white}
mono lxs W --> DT wm b \in lset (produce (batch_op tds) lxs) -->>
(\<forall>t \in set_t b. coll b t = lcoll lxs t + coll tds t) \<and>
set_t b = batch_ts ((map (\<lambda> (t,d). DT t d) tds) @@ lxs) wm
\end{tcblisting}

\begin{tcblisting}{hbox,listing only,listing options={language=isabelle,aboveskip=0pt,belowskip=0pt},size=fbox,boxrule=0pt,frame hidden,arc=0pt,colback=yellow!10!white}
mono lxs W --> (\<forall>t \in set_t tds. \<forall>wm \in W. \<not> t <= wm) -->> mono (produce (batch_op tds) lxs) W
\end{tcblisting}

\end{frame}

% \section{Other shapes}

% \begin{frame}[fragile]
%   \frametitle{Join Operator}
%   \begin{itemize}
%     \item Relation Join
%     \item Use the \is{sum} type in the timestamps to represent two stream as one
%     \item Partial order for the \is{sum}: lefts and rights are incomparable
%     \item Defined using \is{incr_batch_op}
%     \item Soundness, completeness, and monotone and productive preservation
%   \end{itemize}
% \end{frame}

\section{Conclusion}

\begin{frame}
  \frametitle{Conclusion}
  \begin{itemize}
    \item Isabelle/HOL has a good tool set to formalize and reason about stream processing:
          \begin{itemize}
            \item Codatatypes, coinductive predicates, corecursion with friends, reasoning up to friends (congruence),
                  \begin{itemize}
                    \item Coinduction up to congruence principle is automatically derived for codatatypes (but not for coinductive principles)
                  \end{itemize}
          \end{itemize}
    \item Next step: Feedback loop
  \end{itemize}
\end{frame}
% slide about tecnicques used, coinduction principles, friends, reasoning up to friends, generalization using skip_op, combining coinduction and induction, coinduction up to congruence for coinductive predicates, say our paper is more detailed in these topics

\section{Questions, comments and suggestions}

\end{document}
