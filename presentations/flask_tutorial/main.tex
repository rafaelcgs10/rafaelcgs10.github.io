\documentclass[fleqn,aspectratio=169,10pt]{beamer}
\usetheme{Madrid}
\usepackage[T1]{fontenc}

\setbeamertemplate{navigation symbols}{}
\usepackage{listings}
\usepackage{xcolor}

\definecolor{codegreen}{rgb}{0,0.6,0}
\definecolor{codegray}{rgb}{0.5,0.5,0.5}
\definecolor{codepurple}{rgb}{0.58,0,0.82}
\definecolor{backcolour}{rgb}{0.95,0.95,0.92}

\lstdefinestyle{mystyle}{
  backgroundcolor=\color{backcolour},
  commentstyle=\color{codegreen},
  keywordstyle=\color{magenta},
  numberstyle=\tiny\color{codegray},
  stringstyle=\color{codepurple},
  basicstyle=\ttfamily\footnotesize,
  breakatwhitespace=false,
  breaklines=true,
  captionpos=b,
  keepspaces=true,
  numbers=left,
  numbersep=5pt,
  showspaces=false,
  showstringspaces=false,
  showtabs=false,
  tabsize=2
}

\lstset{style=mystyle}

\setbeamertemplate{footline}
{
  \leavevmode%
  \hbox{%
    \begin{beamercolorbox}[wd=.4\paperwidth,ht=2.25ex,dp=1ex,center]{author in head/foot}%
      \usebeamerfont{author in head/foot}\insertshortauthor
    \end{beamercolorbox}%
    \begin{beamercolorbox}[wd=.6\paperwidth,ht=2.25ex,dp=1ex,center]{title in head/foot}%
      \usebeamerfont{title in head/foot}\insertshorttitle\hspace*{3em}
      \insertframenumber{} / \inserttotalframenumber\hspace*{1ex}
    \end{beamercolorbox}}%
  \vskip0pt%
}

\usepackage{wasysym}
\usepackage{fancybox,graphicx,hyperref,url}
\usepackage{booktabs}

\usepackage{listings}
\usepackage{pdfpages}
\usepackage{lstautogobble}

\AtBeginSection[]{
  \begin{frame}[noframenumbering]
    \vfill
    \centering
    \begin{beamercolorbox}[sep=8pt,center,shadow=true,rounded=true]{title}
      \usebeamerfont{title}\insertsectionhead\par%
    \end{beamercolorbox}
    \vfill
  \end{frame}
}

\title[Flask Tutorial]{Flask Tutorial}

\author[Rafael]{Rafael Castro G. Silva}

\date{20/05/2025}

\institute[UCPH]{
  Department of Computer Science \\
  University of Copenhagen}

\begin{document}
\setbeamercovered{invisible}
% \setbeamercovered{dynamic}

\begin{frame}
  \titlepage
\end{frame}

\section{Learning Goals}

\begin{frame}
  \frametitle{Learning Goals}
  \begin{enumerate}
    \item Learn what is Flask
    \item Learn how to bootstrap a Flask web application
    \item Learn some basic Flask concepts
    \item Learn the Model View Control (MVC) pattern
    \item Learn basic Docker
  \end{enumerate}
\end{frame}

\section{Flask Introduction}

\begin{frame}[fragile]
  \frametitle{What is Flask?}
  \begin{itemize}
          \pause
    \item Flask is a micro framework for web applications
          \pause
          \begin{itemize}
            \item Framework: a set of modules and libraries that provide basic functionalities \\
                  May or may nor enforce some standards (e.g. directory structures, MVC pattern, etc)
            \item Micro: enforces very little
            \item In/for Python
          \end{itemize}
  \end{itemize}
  \pause
  \begin{block}{Other frameworks out there}
    \begin{itemize}
      \item Ruby on Rails
      \item Django (Python)
      \item Java Spring
      \item Symfony (PHP)
    \end{itemize}
  \end{block}
\end{frame}

\begin{frame}[fragile]
  \frametitle{Who uses Flask?}
          \pause
  \begin{figure}[]
    \centering
    \includegraphics[width=0.4\textwidth]{companies}
  \end{figure}
\end{frame}

\begin{frame}
  \frametitle{What do I need know?}
  \begin{itemize}
          \pause
    \item Very basic Python
          \begin{itemize}
            \item Variables, functions, indentation, lists, maps (dictionaries), if-then-else, for loops, import libraries
          \end{itemize}
    \item Terminal commands: \texttt{ls}, \texttt{cd}, \texttt{pwd}...
    \item Basic HTML
    \item CSS (optional)
    \item SQL
  \end{itemize}
\end{frame}

\section{Flask Concepts}
\begin{frame}
  \frametitle{The app variable}

\end{frame}

\begin{frame}
  \frametitle{Routing}
  \begin{itemize}
    \item Python decorator
    \item GET POST
  \end{itemize}

\end{frame}

\begin{frame}[fragile]
  \frametitle{Templates}
\begin{lstlisting}[language=Python]
import numpy as np

def incmatrix(genl1,genl2):
    m = len(genl1)
    n = len(genl2)
    M = None #to become the incidence matrix
    VT = np.zeros((n*m,1), int)  #dummy variable

    #compute the bitwise xor matrix
    M1 = bitxormatrix(genl1)
    M2 = np.triu(bitxormatrix(genl2),1)

    for i in range(m-1):
        for j in range(i+1, m):
            [r,c] = np.where(M2 == M1[i,j])
            for k in range(len(r)):
                VT[(i)*n + r[k]] = 1;
                VT[(i)*n + c[k]] = 1;
                VT[(j)*n + r[k]] = 1;
                VT[(j)*n + c[k]] = 1;

                if M is None:
                    M = np.copy(VT)
                else:
                    M = np.concatenate((M, VT), 1)

                VT = np.zeros((n*m,1), int)

    return M
\end{lstlisting}
\end{frame}

\begin{frame}
  \frametitle{Render Template}
  \begin{itemize}
    \item Python decorator
  \end{itemize}

\end{frame}

\section{MVC}

\section{Let's write our web apps!}

\end{document}
