% Created 2020-07-22 Wed 12:03
% Intended LaTeX compiler: pdflatex
\documentclass[presentation]{beamer}
\usepackage[utf8]{inputenc}
\usepackage[T1]{fontenc}
\usepackage{graphicx}
\usepackage{grffile}
\usepackage{longtable}
\usepackage{wrapfig}
\usepackage{rotating}
\usepackage[normalem]{ulem}
\usepackage{amsmath}
\usepackage{textcomp}
\usepackage{amssymb}
\usepackage{capt-of}
\usepackage{hyperref}
\usepackage{graphicx, hyperref, url}
\mode<beamer>{\usetheme{Madrid}}
\usetheme{default}
\author{Rafael Castro - rafaelcgs10.github.io/coq}
\date{22/07/2020}
\title{Formal Methods in the Real World}
\hypersetup{
 pdfauthor={Rafael Castro - rafaelcgs10.github.io/coq},
 pdftitle={Formal Methods in the Real World},
 pdfkeywords={},
 pdfsubject={},
 pdfcreator={Emacs 26.3 (Org mode 9.1.9)}, 
 pdflang={Portuguese}}
\begin{document}

\maketitle

\section{Introdução}
\label{sec:orgfc9fbc9}

\begin{frame}[label={sec:orgd122416}]{Softwares causam desastres}
\begin{itemize}
\item Therac-25: máquina de radioterapia. 6 pessoas morreram.
\item Ariane 5: Foguete fez curso errado e se auto-destruiu. Colocaram um valore de 64bits num variável de 16bits.
\item Knight Capital Group: perdeu 460 milhões de dólares no mercado de ações americano.
\end{itemize}
\end{frame}

\begin{frame}[label={sec:org6a0a6b4}]{Métodos formais}
\begin{itemize}
\item Engenharia de Software
\item Confiança de software:
\begin{enumerate}
\item Especificação: modelos/lógicas/tipos
\item Desenvolvimento: baseado na espeficiação
\item Verificação: mostrar que o desenvolvimento atende a especificação
\end{enumerate}
\end{itemize}
\end{frame}

\begin{frame}[label={sec:org67cc214}]{O que é um software correto?}
\begin{itemize}
\item Navegador
\item Software de usina nuclear
\item Compilador
\item Linguagem de Programação (uma linguagem também segue uma especificação formal)
\end{itemize}
\end{frame}

\section{Os Métodos}
\label{sec:org3705e81}

\begin{frame}[label={sec:orgf1ed021}]{Confiança em software}
\begin{itemize}
\item Testes vs Verficiação formal
\item Para o que cada um serve?
\item "Testes não provam a ausência de bugs, somente a presença deles"
\item Provas substituem testes?
\end{itemize}
\end{frame}

\begin{frame}[label={sec:org49c86fd}]{Verificações automáticas}
\begin{itemize}
\item Verificação automática por sistemas de tipos
\item Analisadores estáticos (linters) - Rubocop não é linter
\item Rust: Tipos lineares, sem mutabilidade, facilita a concorrência
\item Haskell: Sistema de tipos DM, sem efeitos colaterais
\end{itemize}
\end{frame}

\begin{frame}[label={sec:orgac8c9f3}]{Model Checking}
\begin{itemize}
\item (não manjo)
\item Redes de Petri, Cadeias de Markov, Sistema de Eventos (modelos finitos)
\item Muito utilizado na verificação de hardware
\item Definina em modelo matemático para especificar o comportamento de alto nível
\item O model check realiza verificações sobre o modelo e identifica problemas/estados indesejados
\item TLA+
\end{itemize}
\end{frame}

\begin{frame}[fragile,label={sec:orga608c83}]{Linguagens com Sistemas de Tipos Dependentes}
 \begin{itemize}
\item O que são sistemas de tipos dependentes
\item Idris, Agda, Epigram, F*
\end{itemize}
\begin{verbatim}
app : Vect n a -> Vect m a -> Vect (n + m) a
app Nil       ys = ys
app (x :: xs) ys = x :: app xs ys
\end{verbatim}
\end{frame}

\begin{frame}[fragile,label={sec:org2acab98}]{Assistentes de Provas}
 \begin{itemize}
\item São ferramentas que permitem o desenvolvimento de provas matemáticas
\item Programas verificados podem ser extraídos
\item Coq, Isabelle, Twelf
\end{itemize}
\begin{verbatim}
Theorem plus_id_example : forall n m:nat,
  n = m -> n + n = m + m.
Proof.
  intros n m.
  intros H.
  rewrite -> H.
  reflexivity.
Qed.
\end{verbatim}
\end{frame}

\section{Exemplos}
\label{sec:org0ddce3a}

\begin{frame}[label={sec:org69d1e04}]{Compcert}
\begin{itemize}
\item Compilador de C sem bugs!
\item Desenvolvido em Coq
\item \url{http://compcert.inria.fr/}
\item \url{https://www.absint.com/compcert/index.htm}
\end{itemize}
\end{frame}

\begin{frame}[label={sec:orgf85fc76}]{Sel4}
\begin{itemize}
\item Micro-Kernel sem (certas) falhas de segurança
\item Garante o isolamento entre aplicações do sistema
\item \url{https://sel4.systems/}
\item Provas em Isabelle
\end{itemize}
\end{frame}

\begin{frame}[label={sec:org2d579af}]{Linguagens para BlockChain/SmartContracts}
\begin{itemize}
\item É importante garantir que o programa compute dentro de uma expectativa de tempo
\item Garantir questões de segurança: o contrato não pode ser quebrado
\end{itemize}
\end{frame}

\begin{frame}[label={sec:org16916b1}]{Projeto DeepSpec}
\begin{itemize}
\item \url{https://deepspec.org/main}
\end{itemize}
\end{frame}
\end{document}
