% Created 2021-07-13 ter 17:37
% Intended LaTeX compiler: pdflatex
\documentclass[presentation]{beamer}
\usepackage[utf8]{inputenc}
\usepackage[T1]{fontenc}
\usepackage{graphicx}
\usepackage{grffile}
\usepackage{longtable}
\usepackage{wrapfig}
\usepackage{rotating}
\usepackage[normalem]{ulem}
\usepackage{amsmath}
\usepackage{textcomp}
\usepackage{amssymb}
\usepackage{capt-of}
\usepackage{hyperref}
\usepackage{graphicx, hyperref, url}
\mode<beamer>{\usetheme{Madrid}}
\usetheme{default}
\author{Rafael Castro - rafaelcgs10.github.io}
\date{14/07/2021}
\title{BBL sobre Nix}
\hypersetup{
 pdfauthor={Rafael Castro - rafaelcgs10.github.io},
 pdftitle={BBL sobre Nix},
 pdfkeywords={},
 pdfsubject={},
 pdfcreator={Emacs 27.2 (Org mode 9.4.6)}, 
 pdflang={Portuguese}}
\begin{document}

\maketitle

\section{Introdução}
\label{sec:orgd7ee997}

\begin{frame}[label={sec:org38e288e}]{O que é cada coisa?}
\begin{itemize}
\item Nix: Um gerenciador de pacotes \alert{puramente funcional}
\begin{itemize}
\item Nix Expression Language: Uma linguagem pura, funcional e lazy
\item NixOS: Uma distro Linux feita envolta do Nix
\item Home Manager: Um Manager (gerenciador) para espaço de usuário envolta do Nix
\end{itemize}
\end{itemize}
\end{frame}

\section{Show of:}
\label{sec:orge2bd539}

\begin{frame}[label={sec:orgdd40639}]{Show of:}
\begin{itemize}
\item Instalando o meu ambiente numa VM de forma semi-automática
\begin{center}
\includegraphics[width=.9\linewidth]{./meme1.jpg}
\end{center}
\end{itemize}
\end{frame}

\section{Como usar Nix?}
\label{sec:org64c53cb}

\begin{frame}[label={sec:org4413226}]{Como usar Nix?}
\begin{itemize}
\item Basta instalar e usar!
\item Também podemos ter arquivos .nix para definirmos nossos shells.
\end{itemize}
\end{frame}

\section{Como usar Home Manager?}
\label{sec:org7bab2b0}

\begin{frame}[label={sec:orgc1a0a1c}]{Como usar Home Manager?}
\begin{itemize}
\item Basta editar os arquivos .nix
\item O que colocar aqui?
\begin{itemize}
\item Tudo que for relativo a algo específico de um usuário. Ex: vim.
\end{itemize}
\end{itemize}
\end{frame}

\section{Como usar NixOS?}
\label{sec:org543318b}

\begin{frame}[label={sec:org3ca2df8}]{Como usar NixOS?}
\begin{itemize}
\item Basta editar os arquivos .nix em /etc/nix!
\end{itemize}
\end{frame}

\section{Nix vs Docker}
\label{sec:org277f354}
\begin{frame}[label={sec:org4530758}]{Nix vs Docker}
\begin{itemize}
\item Nix não é um substituto para o Docker, ele faz coisas a mais e coisas a menos.
\item Coisas a mais: gerenciamento de dependências do sistema
\item Coisas a menos: não é um ambiente de container, não gerencia recurso do sistema\ldots{}
\end{itemize}
\end{frame}

\section{Vantagems e desvantagens do NixOS}
\label{sec:org7757f4a}
\begin{frame}[label={sec:orgd0d84f1}]{Vantagems e desvantagens do NixOS}
\begin{itemize}
\item Vantagens:
\begin{enumerate}
\item Seu sistema é reproduzível em qualquer lugar
\item Declarativo
\item Modular
\item Facilita litar diferentes versões de pacotes
\item Versionamento
\end{enumerate}
\item Desvantagens:
\begin{enumerate}
\item É uma distro pouco usada, comunidade pequena, pouca documentação, mais problemas do que pessoas
\item Usa mais espaço
\item Usa bastante CPU para instalar pacotes
\item Curva de apredizado pesada
\end{enumerate}
\end{itemize}
\end{frame}
\end{document}
